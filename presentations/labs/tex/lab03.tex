\documentclass[aspectratio=169,12pt]{beamer}
\usepackage[utf8]{inputenc}
\usepackage[T1]{fontenc}
\usepackage{graphicx}
\usepackage{xcolor}
\usepackage{hyperref}
\usepackage{array}

% ===== Theme =====
\usetheme{Madrid}

% ===== Colors =====
\definecolor{spublue}{RGB}{0,102,179}
\definecolor{spulight}{RGB}{230,242,255}

% ===== Beamer Colors =====
\setbeamercolor{palette primary}{bg=spublue,fg=white}
\setbeamercolor{palette secondary}{bg=spublue!80,fg=white}
\setbeamercolor{palette tertiary}{bg=spublue!60,fg=white}
\setbeamercolor{palette quaternary}{bg=spublue,fg=white}
\setbeamercolor{structure}{fg=spublue}
\setbeamercolor{frametitle}{bg=spublue,fg=white}

% ===== Title Page - White text on Blue =====
\setbeamercolor{titlelike}{parent=palette primary}
\setbeamercolor{title}{fg=white,bg=spublue}
\setbeamercolor{subtitle}{fg=white}
\setbeamercolor{author}{fg=black}
\setbeamercolor{institute}{fg=gray}
\setbeamercolor{date}{fg=spublue}

% ===== Custom Title Page =====
\setbeamertemplate{title page}{
    \begin{beamercolorbox}[wd=\paperwidth,ht=2.5cm,dp=0.5cm,center]{palette primary}
        \usebeamerfont{title}\usebeamercolor[fg]{title}\inserttitle\\[0.3cm]
        \usebeamerfont{subtitle}\usebeamercolor[fg]{subtitle}\insertsubtitle
    \end{beamercolorbox}
    \vspace{0.5cm}
    \begin{center}
        \usebeamerfont{author}\usebeamercolor[fg]{author}\insertauthor\\[0.3cm]
        \usebeamerfont{institute}\usebeamercolor[fg]{institute}\insertinstitute\\[0.3cm]
        \usebeamerfont{date}\usebeamercolor[fg]{date}\insertdate
    \end{center}
}

% ===== Remove navigation =====
\setbeamertemplate{navigation symbols}{}
\setbeamertemplate{footline}[frame number]

% ===== Code style =====
\usepackage{listings}
\lstset{
    basicstyle=\ttfamily\small,
    backgroundcolor=\color{gray!10},
    frame=single,
    breaklines=true,
    columns=fullflexible
}

% ===== Title =====
\title{Lab 3: FastAPI}
\subtitle{CSI403 - Full Stack Development}
\author{Faculty of Information Technology}
\institute{Sripatum University}
\date{Weight: 3.75\%}

\begin{document}

% Title slide
\begin{frame}[plain]
\titlepage
\end{frame}

% Objectives
\begin{frame}{Objectives}
\begin{itemize}
    \item[$\checkmark$] Study api.py from Generic-RAG
    \item[$\checkmark$] Run FastAPI server
    \item[$\checkmark$] Test API endpoints
\end{itemize}
\vfill
\textbf{Repository:} \url{https://github.com/amornpan/Generic-RAG}
\end{frame}

% Task 1
\begin{frame}{Task 1: Study api.py}
Review the structure of \texttt{api.py}:
\vfill
\begin{itemize}
    \item \textbf{FastAPI app} - Application instance
    \item \textbf{Pydantic models} - Request/Response schemas
    \item \textbf{Endpoints:}
    \begin{itemize}
        \item \texttt{/health} - Health check
        \item \texttt{/search} - Search documents
        \item \texttt{/query} - RAG query
    \end{itemize}
\end{itemize}
\end{frame}

% Task 2
\begin{frame}[fragile]{Task 2: Run Server}
\begin{lstlisting}[language=bash]
cd Generic-RAG
conda activate rag_env
python api.py
\end{lstlisting}
\vfill
\begin{center}
\fcolorbox{spublue}{spulight}{\parbox{0.8\textwidth}{
\centering
Server runs on \texttt{http://localhost:9000}
}}
\end{center}
\end{frame}

% Task 3
\begin{frame}[fragile]{Task 3: Test Endpoints}
\textbf{Health check:}
\begin{lstlisting}[language=bash]
curl http://localhost:9000/health
\end{lstlisting}
\vfill
\textbf{Search:}
\begin{lstlisting}[language=bash]
curl -X POST http://localhost:9000/search \
  -H "Content-Type: application/json" \
  -d '{"query": "test", "top_k": 5}'
\end{lstlisting}
\end{frame}

% Task 4
\begin{frame}{Task 4: Access Swagger Docs}
\begin{center}
Open in browser:\\[1em]
\Large\textcolor{spublue}{\url{http://localhost:9000/docs}}
\end{center}
\vfill
\begin{itemize}
    \item Interactive API documentation
    \item Test endpoints directly in browser
    \item View request/response schemas
\end{itemize}
\end{frame}

% Task 5
\begin{frame}[fragile]{Task 5: Add New Endpoint}
Add \texttt{/documents} endpoint to list all documents:
\begin{lstlisting}[language=Python]
@app.get("/documents")
async def list_documents():
    # Query OpenSearch for all documents
    response = client.search(
        index="documents",
        body={"query": {"match_all": {}}}
    )
    return {"documents": response["hits"]["hits"]}
\end{lstlisting}
\end{frame}

% Deliverables
\begin{frame}{Deliverables}
\begin{center}
\begin{tabular}{|l|c|}
\hline
\textbf{Item} & \textbf{Check} \\
\hline
Server running & $\square$ \\
Endpoints tested & $\square$ \\
New endpoint added & $\square$ \\
Screenshots & $\square$ \\
\hline
\end{tabular}
\end{center}
\vfill
\begin{center}
\Large\textcolor{red}{\textbf{Deadline: Sunday 23:59}}
\end{center}
\end{frame}

\end{document}
