\documentclass[aspectratio=169,12pt]{beamer}
\usepackage[utf8]{inputenc}
\usepackage[T1]{fontenc}
\usepackage{graphicx}
\usepackage{xcolor}
\usepackage{hyperref}
\usepackage{array}

% ===== Theme =====
\usetheme{Madrid}

% ===== Colors =====
\definecolor{spublue}{RGB}{0,102,179}
\definecolor{spulight}{RGB}{230,242,255}

% ===== Beamer Colors =====
\setbeamercolor{palette primary}{bg=spublue,fg=white}
\setbeamercolor{palette secondary}{bg=spublue!80,fg=white}
\setbeamercolor{palette tertiary}{bg=spublue!60,fg=white}
\setbeamercolor{palette quaternary}{bg=spublue,fg=white}
\setbeamercolor{structure}{fg=spublue}
\setbeamercolor{frametitle}{bg=spublue,fg=white}

% ===== Title Page - White text on Blue =====
\setbeamercolor{titlelike}{parent=palette primary}
\setbeamercolor{title}{fg=white,bg=spublue}
\setbeamercolor{subtitle}{fg=white}
\setbeamercolor{author}{fg=black}
\setbeamercolor{institute}{fg=gray}
\setbeamercolor{date}{fg=spublue}

% ===== Custom Title Page =====
\setbeamertemplate{title page}{
    \begin{beamercolorbox}[wd=\paperwidth,ht=2.5cm,dp=0.5cm,center]{palette primary}
        \usebeamerfont{title}\usebeamercolor[fg]{title}\inserttitle\\[0.3cm]
        \usebeamerfont{subtitle}\usebeamercolor[fg]{subtitle}\insertsubtitle
    \end{beamercolorbox}
    \vspace{0.5cm}
    \begin{center}
        \usebeamerfont{author}\usebeamercolor[fg]{author}\insertauthor\\[0.3cm]
        \usebeamerfont{institute}\usebeamercolor[fg]{institute}\insertinstitute\\[0.3cm]
        \usebeamerfont{date}\usebeamercolor[fg]{date}\insertdate
    \end{center}
}

% ===== Remove navigation =====
\setbeamertemplate{navigation symbols}{}
\setbeamertemplate{footline}[frame number]

% ===== Code style =====
\usepackage{listings}
\lstset{
    basicstyle=\ttfamily\small,
    backgroundcolor=\color{gray!10},
    frame=single,
    breaklines=true,
    columns=fullflexible
}

% ===== Title =====
\title{Lab 1: Git + Python Fundamentals}
\subtitle{CSI403 - Full Stack Development}
\author{Faculty of Information Technology}
\institute{Sripatum University}
\date{Weight: 3.75\%}

\begin{document}

% Title slide
\begin{frame}[plain]
\titlepage
\end{frame}

% Objectives
\begin{frame}{Objectives}
\begin{itemize}
    \item[$\checkmark$] Fork and clone Generic-RAG repository
    \item[$\checkmark$] Practice Git workflow
    \item[$\checkmark$] Review Python fundamentals
\end{itemize}
\vfill
\textbf{Repository:} \url{https://github.com/amornpan/Generic-RAG}
\end{frame}

% Task 1
\begin{frame}{Task 1: Fork Repository}
\begin{enumerate}
    \item Go to \textcolor{spublue}{\url{https://github.com/amornpan/Generic-RAG}}
    \item Click \textbf{``Fork''} button (top right)
    \item You now have your own copy!
\end{enumerate}
\vfill
\begin{center}
\fcolorbox{spublue}{spulight}{\parbox{0.8\textwidth}{
\centering
\textbf{Tip:} Forking creates a personal copy of the repository\\that you can modify without affecting the original.
}}
\end{center}
\end{frame}

% Task 2
\begin{frame}[fragile]{Task 2: Clone to Local}
\begin{lstlisting}[language=bash]
git clone https://github.com/YOUR_USERNAME/Generic-RAG.git
cd Generic-RAG
\end{lstlisting}
\vfill
\textbf{Replace} \texttt{YOUR\_USERNAME} with your GitHub username.
\end{frame}

% Task 3
\begin{frame}[fragile]{Task 3: Setup Environment}
\begin{lstlisting}[language=bash]
# Create conda environment
conda create -n rag_env python=3.10 -y

# Activate environment
conda activate rag_env

# Install dependencies
pip install -r requirements.txt
\end{lstlisting}
\end{frame}

% Task 4
\begin{frame}[fragile]{Task 4: Create Feature Branch}
\begin{lstlisting}[language=bash]
git checkout -b feature/lab01-python
\end{lstlisting}
\vfill
\begin{center}
\fcolorbox{spublue}{spulight}{\parbox{0.8\textwidth}{
\centering
\textbf{Git Flow:} Always create a feature branch\\for new work instead of committing to main.
}}
\end{center}
\end{frame}

% Task 5
\begin{frame}[fragile]{Task 5: Write Python OOP Example}
Create \texttt{lab01\_example.py}:
\begin{lstlisting}[language=Python]
class Document:
    def __init__(self, title: str, content: str):
        self.title = title
        self.content = content
    
    def get_summary(self) -> str:
        return self.content[:100] + "..."

# Test
doc = Document("Test", "This is a test document...")
print(doc.get_summary())
\end{lstlisting}
\end{frame}

% Task 6
\begin{frame}[fragile]{Task 6: Commit and Push}
\begin{lstlisting}[language=bash]
# Stage all changes
git add .

# Commit with message
git commit -m "Lab 1: Add Python OOP example"

# Push to remote
git push origin feature/lab01-python
\end{lstlisting}
\end{frame}

% Task 7
\begin{frame}{Task 7: Create Pull Request}
\begin{enumerate}
    \item Go to your repository on GitHub
    \item Click \textbf{``Compare \& pull request''} button
    \item Add description and create PR
    \item \textbf{Take screenshot} of your PR
\end{enumerate}
\end{frame}

% Deliverables
\begin{frame}{Deliverables}
\begin{center}
\begin{tabular}{|l|c|}
\hline
\textbf{Item} & \textbf{Check} \\
\hline
Forked repository & $\square$ \\
Feature branch created & $\square$ \\
Python file committed & $\square$ \\
Pull Request created & $\square$ \\
Screenshot of PR & $\square$ \\
\hline
\end{tabular}
\end{center}
\vfill
\begin{center}
\Large\textcolor{red}{\textbf{Deadline: Sunday 23:59}}
\end{center}
\end{frame}

\end{document}
