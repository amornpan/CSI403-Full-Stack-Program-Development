\documentclass[12pt,a4paper]{article}

% ===== Basic Packages Only =====
\usepackage[utf8]{inputenc}
\usepackage[T1]{fontenc}
\usepackage[margin=2.5cm]{geometry}
\usepackage{graphicx}
\usepackage{xcolor}
\usepackage{hyperref}
\usepackage{array}
\usepackage{verbatim}

% ===== Colors =====
\definecolor{spublue}{RGB}{0,102,179}
\definecolor{codebg}{RGB}{245,245,240}

% ===== Simple Code Environment =====
\newenvironment{code}{\verbatim}{\endverbatim}

% ===== Document Info =====
\title{
    \vspace{-1cm}
    \textcolor{spublue}{\textbf{Full-Stack RAG with Local LLM}}\\[0.5cm]
    \large Laboratory Manual\\[0.3cm]
    \normalsize CSI403 - Full Stack Development\\
    Faculty of Information Technology, Sripatum University
}
\author{}
\date{Academic Year 2568}

\begin{document}

\maketitle
\thispagestyle{empty}
\newpage
\tableofcontents
\newpage

% ============================================================
% LAB 1
% ============================================================
\section{Lab 1: Git + Python Fundamentals (3.75\%)}

\subsection{Objectives}
\begin{itemize}
    \item Fork and clone Generic-RAG repository
    \item Practice Git workflow
    \item Review Python fundamentals
\end{itemize}

\subsection{Tasks}

\subsubsection{Task 1: Fork Repository}
\begin{enumerate}
    \item Go to https://github.com/amornpan/Generic-RAG
    \item Click ``Fork'' button
    \item You now have your own copy
\end{enumerate}

\subsubsection{Task 2: Clone to Local}
\begin{verbatim}
git clone https://github.com/YOUR_USERNAME/Generic-RAG.git
cd Generic-RAG
\end{verbatim}

\subsubsection{Task 3: Setup Environment}
\begin{verbatim}
conda create -n rag_env python=3.10 -y
conda activate rag_env
pip install -r requirements.txt
\end{verbatim}

\subsubsection{Task 4: Create Feature Branch}
\begin{verbatim}
git checkout -b feature/lab01-python
\end{verbatim}

\subsubsection{Task 5: Write Python OOP Example}
Create \texttt{lab01\_example.py}:
\begin{verbatim}
class Document:
    def __init__(self, title: str, content: str):
        self.title = title
        self.content = content
    
    def get_summary(self) -> str:
        return self.content[:100] + "..."

# Test
doc = Document("Test", "This is a test document...")
print(doc.get_summary())
\end{verbatim}

\subsubsection{Task 6: Commit and Push}
\begin{verbatim}
git add .
git commit -m "Lab 1: Add Python OOP example"
git push origin feature/lab01-python
\end{verbatim}

\subsubsection{Task 7: Create Pull Request}
\begin{enumerate}
    \item Go to GitHub
    \item Click ``Compare \& pull request''
    \item Create PR
\end{enumerate}

\subsection{Deliverables}
\begin{itemize}
    \item[$\square$] Forked repository
    \item[$\square$] Feature branch created
    \item[$\square$] Python file committed
    \item[$\square$] Pull Request created
    \item[$\square$] Screenshot of PR
\end{itemize}

\textbf{Deadline:} Sunday 23:59

\newpage

% ============================================================
% LAB 2
% ============================================================
\section{Lab 2: Docker + OpenSearch (3.75\%)}

\subsection{Objectives}
\begin{itemize}
    \item Install Docker Desktop
    \item Run OpenSearch container
    \item Configure Hybrid Search Pipeline
\end{itemize}

\subsection{Tasks}

\subsubsection{Task 1: Install Docker Desktop}
\begin{enumerate}
    \item Download from https://docker.com
    \item Install and start
    \item Verify: \texttt{docker --version}
\end{enumerate}

\subsubsection{Task 2: Run OpenSearch}
\begin{verbatim}
docker run -d --name opensearch-node \
  -p 9200:9200 -p 9600:9600 \
  -e "discovery.type=single-node" \
  -e "DISABLE_SECURITY_PLUGIN=true" \
  opensearchproject/opensearch:2.11.1
\end{verbatim}

\subsubsection{Task 3: Verify}
\begin{verbatim}
curl http://localhost:9200
\end{verbatim}

\subsubsection{Task 4: Setup Hybrid Search Pipeline}
\begin{verbatim}
curl -X PUT "localhost:9200/_search/pipeline/hybrid-search-pipeline" \
  -H "Content-Type: application/json" \
  -d '{
    "phase_results_processors": [{
      "normalization-processor": {
        "normalization": {"technique": "min_max"},
        "combination": {
          "technique": "arithmetic_mean",
          "parameters": {"weights": [0.3, 0.7]}
        }
      }
    }]
  }'
\end{verbatim}

\subsubsection{Task 5: Check Health}
\begin{verbatim}
curl http://localhost:9200/_cluster/health?pretty
\end{verbatim}

\subsection{Deliverables}
\begin{itemize}
    \item[$\square$] Docker installed
    \item[$\square$] OpenSearch running
    \item[$\square$] Pipeline created
    \item[$\square$] Screenshots
\end{itemize}

\textbf{Deadline:} Sunday 23:59

\newpage

% ============================================================
% LAB 3
% ============================================================
\section{Lab 3: FastAPI (3.75\%)}

\subsection{Objectives}
\begin{itemize}
    \item Study api.py from Generic-RAG
    \item Run FastAPI server
    \item Test API endpoints
\end{itemize}

\subsection{Tasks}

\subsubsection{Task 1: Study api.py}
Review the structure:
\begin{itemize}
    \item FastAPI app
    \item Pydantic models
    \item Endpoints: /health, /search, /query
\end{itemize}

\subsubsection{Task 2: Run Server}
\begin{verbatim}
cd Generic-RAG
conda activate rag_env
python api.py
\end{verbatim}

\subsubsection{Task 3: Test Endpoints}
\begin{verbatim}
# Health check
curl http://localhost:9000/health

# Search
curl -X POST http://localhost:9000/search \
  -H "Content-Type: application/json" \
  -d '{"query": "test", "top_k": 5}'
\end{verbatim}

\subsubsection{Task 4: Access Swagger Docs}
Open: http://localhost:9000/docs

\subsubsection{Task 5: Add New Endpoint}
Add \texttt{/documents} endpoint to list all documents.

\subsection{Repository}
https://github.com/amornpan/Generic-RAG

\subsection{Deliverables}
\begin{itemize}
    \item[$\square$] Server running
    \item[$\square$] Endpoints tested
    \item[$\square$] New endpoint added
    \item[$\square$] Screenshots
\end{itemize}

\textbf{Deadline:} Sunday 23:59

\newpage

% ============================================================
% LAB 4
% ============================================================
\section{Lab 4: OpenSearch Integration (3.75\%)}

\subsection{Objectives}
\begin{itemize}
    \item Connect to OpenSearch from Python
    \item Implement hybrid search
    \item Index and search documents
\end{itemize}

\subsection{Tasks}

\subsubsection{Task 1: Connect to OpenSearch}
\begin{verbatim}
from opensearchpy import OpenSearch

client = OpenSearch(
    hosts=[{"host": "localhost", "port": 9200}],
    use_ssl=False
)
print(client.info())
\end{verbatim}

\subsubsection{Task 2: Create Index}
\begin{verbatim}
index_body = {
    "settings": {"index": {"knn": True}},
    "mappings": {
        "properties": {
            "content": {"type": "text"},
            "content_vector": {
                "type": "knn_vector",
                "dimension": 1024
            }
        }
    }
}
client.indices.create(index="documents", body=index_body)
\end{verbatim}

\subsubsection{Task 3: Index Documents}
\begin{verbatim}
doc = {
    "content": "Sample document text",
    "content_vector": [0.1, 0.2, ...]  # 1024 dims
}
client.index(index="documents", body=doc)
\end{verbatim}

\subsubsection{Task 4: Implement Search}
Implement hybrid search combining vector and BM25.

\subsection{Repository}
https://github.com/amornpan/Generic-RAG

\subsection{Deliverables}
\begin{itemize}
    \item[$\square$] Connection working
    \item[$\square$] Index created
    \item[$\square$] Documents indexed
    \item[$\square$] Search implemented
\end{itemize}

\textbf{Deadline:} Sunday 23:59

\newpage

% ============================================================
% LAB 5
% ============================================================
\section{Lab 5: Embeddings (3.75\%)}

\subsection{Objectives}
\begin{itemize}
    \item Run embedding.py
    \item Understand chunking
    \item Add new documents
\end{itemize}

\subsection{Tasks}

\subsubsection{Task 1: Study embedding.py}
Review the pipeline:
\begin{enumerate}
    \item Load documents from md\_corpus/
    \item Chunk documents
    \item Create embeddings (bge-m3)
    \item Index to OpenSearch
\end{enumerate}

\subsubsection{Task 2: Run Indexing}
\begin{verbatim}
cd Generic-RAG
conda activate rag_env
python embedding.py
\end{verbatim}

\subsubsection{Task 3: Verify}
\begin{verbatim}
curl http://localhost:9200/documents/_count
\end{verbatim}

\subsubsection{Task 4: Add New Documents}
Add 3 new .md files to \texttt{md\_corpus/} folder.

\subsubsection{Task 5: Re-index}
\begin{verbatim}
python embedding.py
\end{verbatim}

\subsubsection{Task 6: Test Search}
Search for content from your new documents.

\subsection{Repository}
https://github.com/amornpan/Generic-RAG

\subsection{Deliverables}
\begin{itemize}
    \item[$\square$] Indexing completed
    \item[$\square$] Documents verified
    \item[$\square$] New documents added
    \item[$\square$] Search tested
\end{itemize}

\textbf{Deadline:} Sunday 23:59

\newpage

% ============================================================
% LAB 6
% ============================================================
\section{Lab 6: RAG + Local LLM + Streamlit (3.75\%)}

\subsection{Objectives}
\begin{itemize}
    \item Setup Ollama
    \item Complete RAG pipeline
    \item Run Streamlit UI
\end{itemize}

\subsection{Tasks}

\subsubsection{Task 1: Install Ollama}
\begin{enumerate}
    \item Download from https://ollama.ai
    \item Install and run
\end{enumerate}

\subsubsection{Task 2: Pull Model}
\begin{verbatim}
ollama pull qwen2.5:7b
\end{verbatim}

\subsubsection{Task 3: Test LLM}
\begin{verbatim}
ollama run qwen2.5:7b
# Type: "What is RAG?"
\end{verbatim}

\subsubsection{Task 4: Run API with LLM}
\begin{verbatim}
cd Generic-RAG
python api.py
\end{verbatim}

\subsubsection{Task 5: Run Streamlit}
\begin{verbatim}
streamlit run app.py
\end{verbatim}

\subsubsection{Task 6: Test Complete Flow}
\begin{enumerate}
    \item Open http://localhost:8501
    \item Ask questions
    \item Verify RAG responses
\end{enumerate}

\subsubsection{Task 7: Modify Prompt}
Edit the prompt template in api.py.

\subsection{Repository}
https://github.com/amornpan/Generic-RAG

\subsection{Deliverables}
\begin{itemize}
    \item[$\square$] Ollama running
    \item[$\square$] LLM tested
    \item[$\square$] Complete RAG working
    \item[$\square$] Streamlit UI working
\end{itemize}

\textbf{Deadline:} Sunday 23:59

\newpage

% ============================================================
% LAB 7
% ============================================================
\section{Lab 7: Docker Compose (3.75\%)}

\subsection{Objectives}
\begin{itemize}
    \item Clone Advanced-RAG-Docker
    \item Run 6 services with Docker Compose
    \item Verify all services
\end{itemize}

\subsection{Tasks}

\subsubsection{Task 1: Clone Repository}
\begin{verbatim}
git clone https://github.com/amornpan/Advanced-RAG-Docker.git
cd Advanced-RAG-Docker
\end{verbatim}

\subsubsection{Task 2: Study docker-compose.yml}
Review the 6 services:

\begin{center}
\begin{tabular}{|l|l|}
\hline
\textbf{Service} & \textbf{Port} \\
\hline
opensearch\_container & 9200 \\
embedding\_container & - \\
search\_api\_container & 8005 \\
backend\_container & 8006 \\
frontend\_container & 8501 \\
ollama\_container & 11434 \\
\hline
\end{tabular}
\end{center}

\subsubsection{Task 3: Start Services}
\begin{verbatim}
docker-compose up -d
\end{verbatim}

\subsubsection{Task 4: Check Status}
\begin{verbatim}
docker-compose ps
\end{verbatim}

\subsubsection{Task 5: View Logs}
\begin{verbatim}
docker-compose logs -f
\end{verbatim}

\subsubsection{Task 6: Test Application}
Open http://localhost:8501

\subsubsection{Task 7: Stop Services}
\begin{verbatim}
docker-compose down
\end{verbatim}

\subsection{Repository}
https://github.com/amornpan/Advanced-RAG-Docker

\subsection{Deliverables}
\begin{itemize}
    \item[$\square$] All 6 services running
    \item[$\square$] Application accessible
    \item[$\square$] Screenshots of docker ps
    \item[$\square$] Document observations
\end{itemize}

\textbf{Deadline:} Sunday 23:59

\newpage

% ============================================================
% LAB 8
% ============================================================
\section{Lab 8: CI/CD + Testing (3.75\%)}

\subsection{Objectives}
\begin{itemize}
    \item Write pytest tests
    \item Create CI/CD pipeline
    \item Setup GitHub Actions
\end{itemize}

\subsection{Tasks}

\subsubsection{Task 1: Write test\_api.py}
\begin{verbatim}
import pytest
from fastapi.testclient import TestClient
from api import app

client = TestClient(app)

def test_health():
    response = client.get("/health")
    assert response.status_code == 200
    assert response.json()["status"] == "healthy"

def test_search():
    response = client.post("/search", 
        json={"query": "test", "top_k": 5})
    assert response.status_code == 200
\end{verbatim}

\subsubsection{Task 2: Run Tests}
\begin{verbatim}
pytest tests/ -v
pytest tests/ --cov=. --cov-report=html
\end{verbatim}

\subsubsection{Task 3: Create Jenkinsfile}
\begin{verbatim}
pipeline {
    agent any
    stages {
        stage('Build') {
            steps { sh 'docker-compose build' }
        }
        stage('Test') {
            steps { sh 'pytest tests/' }
        }
        stage('Deploy') {
            steps { sh 'docker-compose up -d' }
        }
    }
}
\end{verbatim}

\subsubsection{Task 4: Create GitHub Actions}
Create \texttt{.github/workflows/ci.yml}:
\begin{verbatim}
name: CI/CD
on: [push, pull_request]
jobs:
  build:
    runs-on: ubuntu-latest
    steps:
      - uses: actions/checkout@v3
      - run: docker-compose build
      - run: docker-compose up -d
      - run: sleep 60
      - run: curl -f http://localhost:8006/health
\end{verbatim}

\subsection{Repository}
https://github.com/amornpan/Advanced-RAG-Docker

\subsection{Deliverables}
\begin{itemize}
    \item[$\square$] Tests written
    \item[$\square$] Tests passing
    \item[$\square$] Jenkinsfile created
    \item[$\square$] GitHub Actions workflow
    \item[$\square$] Documentation
\end{itemize}

\textbf{Deadline:} Sunday 23:59

\newpage

% ============================================================
% Summary Table
% ============================================================
\section*{Summary: All Labs}
\addcontentsline{toc}{section}{Summary: All Labs}

\begin{center}
\begin{tabular}{|c|l|c|}
\hline
\textbf{Lab} & \textbf{Topic} & \textbf{Weight} \\
\hline
1 & Git + Python Fundamentals & 3.75\% \\
2 & Docker + OpenSearch & 3.75\% \\
3 & FastAPI & 3.75\% \\
4 & OpenSearch Integration & 3.75\% \\
5 & Embeddings & 3.75\% \\
6 & RAG + Local LLM + Streamlit & 3.75\% \\
7 & Docker Compose & 3.75\% \\
8 & CI/CD + Testing & 3.75\% \\
\hline
& \textbf{Total} & \textbf{30\%} \\
\hline
\end{tabular}
\end{center}

\subsection*{Repositories}
\begin{itemize}
    \item Generic-RAG: https://github.com/amornpan/Generic-RAG
    \item Advanced-RAG-Docker: https://github.com/amornpan/Advanced-RAG-Docker
\end{itemize}

\end{document}
