\documentclass[aspectratio=169,12pt]{beamer}
\usepackage[utf8]{inputenc}
\usepackage[T1]{fontenc}
\usepackage{graphicx}
\usepackage{xcolor}
\usepackage{hyperref}
\usepackage{array}

% ===== Theme =====
\usetheme{Madrid}

% ===== Colors =====
\definecolor{spublue}{RGB}{0,102,179}
\definecolor{spulight}{RGB}{230,242,255}

% ===== Beamer Colors =====
\setbeamercolor{palette primary}{bg=spublue,fg=white}
\setbeamercolor{palette secondary}{bg=spublue!80,fg=white}
\setbeamercolor{palette tertiary}{bg=spublue!60,fg=white}
\setbeamercolor{palette quaternary}{bg=spublue,fg=white}
\setbeamercolor{structure}{fg=spublue}
\setbeamercolor{frametitle}{bg=spublue,fg=white}

% ===== Title Page - White text on Blue =====
\setbeamercolor{titlelike}{parent=palette primary}
\setbeamercolor{title}{fg=white,bg=spublue}
\setbeamercolor{subtitle}{fg=white}
\setbeamercolor{author}{fg=black}
\setbeamercolor{institute}{fg=gray}
\setbeamercolor{date}{fg=spublue}

% ===== Custom Title Page =====
\setbeamertemplate{title page}{
    \begin{beamercolorbox}[wd=\paperwidth,ht=2.5cm,dp=0.5cm,center]{palette primary}
        \usebeamerfont{title}\usebeamercolor[fg]{title}\inserttitle\\[0.3cm]
        \usebeamerfont{subtitle}\usebeamercolor[fg]{subtitle}\insertsubtitle
    \end{beamercolorbox}
    \vspace{0.5cm}
    \begin{center}
        \usebeamerfont{author}\usebeamercolor[fg]{author}\insertauthor\\[0.3cm]
        \usebeamerfont{institute}\usebeamercolor[fg]{institute}\insertinstitute\\[0.3cm]
        \usebeamerfont{date}\usebeamercolor[fg]{date}\insertdate
    \end{center}
}

% ===== Remove navigation =====
\setbeamertemplate{navigation symbols}{}
\setbeamertemplate{footline}[frame number]

% ===== Code style =====
\usepackage{listings}
\lstset{
    basicstyle=\ttfamily\small,
    backgroundcolor=\color{gray!10},
    frame=single,
    breaklines=true,
    columns=fullflexible
}

% ===== Title =====
\title{Lab 5: Embeddings}
\subtitle{CSI403 - Full Stack Development}
\author{Faculty of Information Technology}
\institute{Sripatum University}
\date{Weight: 3.75\%}

\begin{document}

% Title slide
\begin{frame}[plain]
\titlepage
\end{frame}

% Objectives
\begin{frame}{Objectives}
\begin{itemize}
    \item[$\checkmark$] Run embedding.py
    \item[$\checkmark$] Understand chunking
    \item[$\checkmark$] Add new documents
\end{itemize}
\vfill
\textbf{Repository:} \url{https://github.com/amornpan/Generic-RAG}
\end{frame}

% Task 1
\begin{frame}{Task 1: Study embedding.py}
Review the pipeline:
\vfill
\begin{enumerate}
    \item \textbf{Load} documents from \texttt{md\_corpus/}
    \item \textbf{Chunk} documents into smaller pieces
    \item \textbf{Create embeddings} using bge-m3 model
    \item \textbf{Index} to OpenSearch
\end{enumerate}
\vfill
\begin{center}
\fcolorbox{spublue}{spulight}{\parbox{0.7\textwidth}{
\centering
\textbf{bge-m3} produces 1024-dimensional vectors\\supporting multilingual text
}}
\end{center}
\end{frame}

% Task 2
\begin{frame}[fragile]{Task 2: Run Indexing}
\begin{lstlisting}[language=bash]
cd Generic-RAG
conda activate rag_env
python embedding.py
\end{lstlisting}
\vfill
This will:
\begin{itemize}
    \item Read all \texttt{.md} files from \texttt{md\_corpus/}
    \item Split into chunks
    \item Generate embeddings
    \item Store in OpenSearch
\end{itemize}
\end{frame}

% Task 3
\begin{frame}[fragile]{Task 3: Verify}
\begin{lstlisting}[language=bash]
curl http://localhost:9200/documents/_count
\end{lstlisting}
\vfill
\textbf{Expected response:}
\begin{lstlisting}[language=bash]
{
  "count": 42,
  "_shards": { "total": 1, "successful": 1 }
}
\end{lstlisting}
\end{frame}

% Task 4-5
\begin{frame}[fragile]{Task 4-5: Add New Documents}
\textbf{Step 1:} Add 3 new \texttt{.md} files to \texttt{md\_corpus/} folder
\vfill
\textbf{Step 2:} Re-index
\begin{lstlisting}[language=bash]
python embedding.py
\end{lstlisting}
\vfill
\textbf{Step 3:} Verify count increased
\begin{lstlisting}[language=bash]
curl http://localhost:9200/documents/_count
\end{lstlisting}
\end{frame}

% Task 6
\begin{frame}{Task 6: Test Search}
Search for content from your new documents:
\vfill
\begin{itemize}
    \item Use the API endpoint \texttt{/search}
    \item Or test via Swagger UI
    \item Verify your new content appears in results
\end{itemize}
\end{frame}

% Deliverables
\begin{frame}{Deliverables}
\begin{center}
\begin{tabular}{|l|c|}
\hline
\textbf{Item} & \textbf{Check} \\
\hline
Indexing completed & $\square$ \\
Documents verified & $\square$ \\
New documents added & $\square$ \\
Search tested & $\square$ \\
\hline
\end{tabular}
\end{center}
\vfill
\begin{center}
\Large\textcolor{red}{\textbf{Deadline: Sunday 23:59}}
\end{center}
\end{frame}

\end{document}
