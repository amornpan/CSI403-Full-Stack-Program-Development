% Week 2: Git + Python Fundamentals
\documentclass[aspectratio=169,11pt]{beamer}
\usetheme{Madrid}
\usecolortheme{default}
\setbeamertemplate{navigation symbols}{}
\usepackage[T1]{fontenc}
\usepackage{booktabs}
\usepackage{listings}
\lstset{basicstyle=\ttfamily\small,breaklines=true}

% === PURPLE THEME ===
\definecolor{maincolor}{RGB}{138,43,226}
\definecolor{accentcolor}{RGB}{186,85,211}
\definecolor{lightpurple}{RGB}{230,190,255}

\setbeamercolor{structure}{fg=maincolor}
\setbeamercolor{palette primary}{bg=maincolor,fg=white}
\setbeamercolor{palette secondary}{bg=accentcolor,fg=white}
\setbeamercolor{palette tertiary}{bg=maincolor,fg=white}
\setbeamercolor{title}{fg=white}
\setbeamercolor{frametitle}{fg=white,bg=maincolor}
\setbeamercolor{block title}{bg=maincolor,fg=white}
\setbeamercolor{block body}{bg=lightpurple!30}
\setbeamercolor{titlelike}{fg=white,bg=maincolor}

\title{Week 2: Git + Python Fundamentals}
\subtitle{Full Stack RAG with Local LLM}
\date{Semester 2/2568}

\begin{document}

\begin{frame}
    \titlepage
\end{frame}

\begin{frame}{Agenda}
    \tableofcontents
\end{frame}

\section{Git Fundamentals}

\begin{frame}{Why Git?}
    \textbf{Version Control System}
    \begin{itemize}
        \item Track changes in your code
        \item Collaborate with others
        \item Revert to previous versions
        \item Branch for new features
    \end{itemize}
\end{frame}

\begin{frame}[fragile]{Git Basic Commands}
    \begin{lstlisting}[language=bash]
# Clone repository
git clone https://github.com/amornpan/Generic-RAG.git

# Check status
git status

# Add and commit
git add .
git commit -m "Add new feature"

# Push to remote
git push origin main
    \end{lstlisting}
\end{frame}

\section{Python Fundamentals}

\begin{frame}[fragile]{Python Data Types}
    \begin{lstlisting}[language=python]
# Type hints
name: str = "RAG System"
count: int = 42
score: float = 0.95
is_active: bool = True

# Collections
items: list = [1, 2, 3]
config: dict = {"model": "bge-m3"}
    \end{lstlisting}
\end{frame}

\begin{frame}[fragile]{Python Classes}
    \begin{lstlisting}[language=python]
class Document:
    def __init__(self, title: str, content: str):
        self.title = title
        self.content = content
    
    def get_summary(self) -> str:
        return self.content[:100] + "..."
    \end{lstlisting}
\end{frame}

\section{Environment Setup}

\begin{frame}[fragile]{Conda Environment}
    \begin{lstlisting}[language=bash]
# Create environment
conda create -n rag_env python=3.10 -y

# Activate
conda activate rag_env

# Install dependencies
pip install -r requirements.txt
    \end{lstlisting}
\end{frame}

\section{Lab 1}

\begin{frame}{Lab 1: Git + Python (3.75\%)}
    \textbf{Tasks:}
    \begin{enumerate}
        \item Fork Generic-RAG repository
        \item Clone to local machine
        \item Create conda environment
        \item Create feature branch
        \item Write Python OOP example
        \item Commit and push
    \end{enumerate}
    \textbf{Deadline: Sunday 23:59}
\end{frame}

\begin{frame}
    \begin{center}
        \Huge Questions?
        \Large See you in Lab 1!
    \end{center}
\end{frame}

\end{document}
